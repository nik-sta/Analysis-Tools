\newpage
\section{Einführung}

\subsection*{Geschichte für die Erläuterung des Begriffs ''Analyse Tool''}
Vor vielen Jahren erhielt Archimedes einen kniffligen Auftrag. König Hieron II. von Syrakus bat ihn herauszufinden, ob seine Krone tatsächlich aus purem Gold bestand.

\par \medskip

Der König hatte einem Goldschmied einen grossen Goldklumpen gegeben, aus dem dieser eine Krone fertigen sollte. Als das Schmuckstück fertig war, versicherte der Schmied dem König, dass die Krone aus reinem Gold war. Doch der Herrscher hegte einen Verdacht: Hatte der Schmied heimlich Silber beigemischt, um einen Teil des wertvolleren Goldes für sich zu behalten?

\par \medskip

Die zündende Idee kam Archimedes der Legende nach, als er in eine randvolle Badewanne stieg und das Wasser überschwappte. Er begriff, dass sein Körper eine ganz bestimmte Menge Wasser verdrängt hatte. Dieser Geistesblitz überwältigte Archimedes angelblich so sehr, dass er vor Freude aus der Wanne sprang und laut ''Heureka, heureka'' geschrieen haben soll. Das ist griechisch und bedeutet ''Ich hab's gefunden!''
\par \medskip
\textbf{Was Archimedes entdeckt hat war ein Analyse Tool, welches in Kombination mit einer simplen Skala, entscheiden konnte, ob die Krone des Königs aus purem Gold ist.}

\subsection*{Bezug zu unserem Thema ''Analyse Tools''}
Bei der Entwicklung von Programmen legen wir einen grossen Wert darauf ''guten'' Code zu schreiben. Wobei wir unter Code \glspl{algorithmus} und \glspl{datenstruktur} verstehen. Aber um solche \glspl{algorithmus} und \glspl{datenstruktur} als ''gut'' bezeichnen zu können, benötigen wir prä­zis Analyse-Werkzeuge. 

\par \medskip

Unser Ziel ist es die Laufzeit eines \gls{algorithmus} und die Operationen auf der \gls{datenstruktur} inklusive der Speicherverwendung chrakteriisieren zu können. Die Laufzeit ist hierbei die zentrale Masseinheit für einen ''guten'' \gls{algorithmus}. Diese Laufzeit ist generell von der zu verarbeitenden Datenmenge abhängig. Jedoch, aber auch von weiteren Faktoren, wie CPU-Belastung, Hardware, Betriebssystem, Programmiersprache, etc \dots Wir konzentrieren uns im Weiteren ausschliesslich auf die Laufzeit des \gls{algorithmus} in Abhängigkeit zur Grösse der Eingabe (Input). Diese Gegebenheit erlaubt es uns, eine handvoll mathematischer Funktionen für die Analyse des \gls{algorithmus} zu verwenden.
$$Funktion(Input) = Laufzeit\ abhängig\ von\ der\ Inputgrösse$$
