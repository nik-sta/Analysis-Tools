\section{Übersicht der Thematik}
Das folgende Mindmap \emph{Abbildung \ref{fig:mindmap-thematik}} soll grob die behandelten und prüfungsrelevanten Themen visualisieren.

%% Beginn des Mindmaps
\begin{figure}[!h]
\centering
	\begin{tikzpicture}
  		\path[mindmap,concept color={rgb:red,1;green,2;blue,5},text=white]
    	node[concept] {Analysis Tools} [clockwise from=-30]
    	child[concept] { 
    		node[concept] {Empirische Analyse}
    	}
    	child[concept, thick] { 
    		node[concept] {Mathematische Funktionen} 
    	}
    	child[concept, thick] { 
    		node[concept] {Big-Oh Notation} 
    	};   
	\end{tikzpicture}
	
\caption{Mindmap über die behandelte Thematik in dieser Arbeit.}
\label{fig:mindmap-thematik}
\end{figure}

Für die Modulprüfung ist es notwendig, die \textbf{O-Abschätzung} durchführen zu können. Als gegeben im Test ist das Zeitverhalten eines Algorithmus anzusehen. In der Klausur vom Frühlingssemester 2015 machte diese Aufgabe \textbf{\nicefrac{1}{9} der Gesamtpunktzahl} aus und ist somit eines der Hauptthemen.