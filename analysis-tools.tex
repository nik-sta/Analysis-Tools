%%%%%%%%%%%%%%%%%%%%%%%%%%%%%%%%%%%%%%
%% Summary of Analysis Tools in Data Structues & Algorithms in Java.
%% =================================================
%% filename: 	analysis-tools.tex
%% created:		20.04.2016
%% -----------------------------------------------------------------
%% (C) 2016 by 	Nikola Stanković
%%				University of Applied Sciences Rapperswil
%%				Switzerland
%%%%%%%%%%%%%%%%%%%%%%%%%%%%%%%%%%%%%%

\documentclass[a4paper,12pt]{scrartcl}

%% Load packages
%% ========
\usepackage[utf8]{inputenc}	% this is needed for umlauts
\usepackage[ngerman]{babel} % this is needed for umlauts
\usepackage[T1]{fontenc}    % this is needed for correct output of umlauts in pdf
\usepackage{lmodern}
\usepackage[]{setspace}
\usepackage{amsmath}
\usepackage{amssymb}
\usepackage{geometry}
\usepackage{graphicx}
\usepackage[colorlinks=true]{hyperref}		% this is needed for the hyper references
\usepackage{nicefrac}		% this is needed for the skewed fractions

%% For the plots
\usepackage{pgfplots}
\pgfplotsset{width=10cm,compat=1.9}

%% Use Helvetica (is sans serif) as default font
\renewcommand{\familydefault}{\sfdefault}
\usepackage{helvet}

\onehalfspacing

%% Glossary
\usepackage[xindy]{glossaries} 
\newglossaryentry{algorithmus}
{
	name={Algorithmus},
	description={ist eine Schritt-für-Schritt Anleitung für die Ausführung einer Aufgabe, bzw. Lösung einer Problemstellung, in einer begrenzten Dauer},
	plural={Algorithmen}	
}

\newglossaryentry{datenstruktur}
{
	name={Datenstruktur},
	description={ist eine systematische Art und Weise, um Daten zu organisieren und darauf zuzugreifen},
	plural={Datenstrukturen}	
}
\makeglossaries

%% For drawing the mindmap
\usepackage{tikz}
\usetikzlibrary{mindmap,trees}

\geometry{papersize={210mm,297mm},total={160mm,240mm},top=31mm,bindingoffset=15mm}

\setlength{\headheight}{20pt}

%% Own variables
%% ========
\newcommand{\documenttitel}{Analyse Tools}
\newcommand{\documentsubtitel}{Algorithmen und Datenstrukturen 1}
\newcommand{\documentauthors}{Nikola Stanković}
\newcommand\HRule{\noindent\rule{\linewidth}{2pt}}

%% PDF meta informations
%% ========
\hypersetup{
	pdfauthor={\documentauthors},%
	pdftitle={\documenttitel},%
	pdfsubject={\documentsubtitel},%
	pdfkeywords={AD1, Analysis-Tools, Big-Oh Notation, HSR},%
	pdfproducer={LaTeX},%
	pdfcreator={pdfLaTeX}
	linktoc=all,
    linkcolor=blue
}

%% For displaying Java code with syntax highlighting
\usepackage{listings}
\usepackage{color}

\definecolor{dkgreen}{rgb}{0,0.6,0}
\definecolor{gray}{rgb}{0.5,0.5,0.5}
\definecolor{mauve}{rgb}{0.58,0,0.82}

\lstset{frame=tb,
  language=Java,
  aboveskip=3mm,
  belowskip=3mm,
  showstringspaces=false,
  columns=flexible,
  basicstyle={\small\ttfamily},
  numbers=none,
  numberstyle=\tiny\color{gray},
  keywordstyle=\color{blue},
  commentstyle=\color{dkgreen},
  stringstyle=\color{mauve},
  breaklines=true,
  breakatwhitespace=true,
  tabsize=3
}

%% For header and footer
\usepackage{calc}
\usepackage{lipsum}
\usepackage{scrpage2}
\usepackage{lastpage} 
\pagestyle{scrheadings}
\clearscrheadfoot
\automark[chapter]{section}
\ohead{\headmark}
\ihead{\documenttitel, Schwerpunkt: Big-Oh Notation}
\setheadsepline[\textwidth+5pt]{0.5pt}
\ofoot[{Seite \thepage\ von \pageref*{LastPage}}]{Seite \thepage\ von \pageref*{LastPage}}

\setlength{\parindent}{0em} 
 
\begin{document}

\author{
  Stanković, Nikola
}

\begin{titlepage}
	\vspace*{2cm}
	\HRule
	\vspace*{5pt}
	\begin{flushright}
	{\Huge \textbf{\documenttitel, \\ Schwerpunkt: Big-Oh Notation} \\ \vspace*{25pt} \Large \documentsubtitel}
	\end{flushright}
	\vspace*{5pt}
	\HRule
	\begin{flushright}
	\vspace{20pt}
	\LARGE
	\documentauthors
	\end{flushright}
	
	\vfill	
	
	\begin{center}
		\includegraphics[width=0.5\textwidth]{images/HSR_Logo_CMYK.jpg}\par\vspace{1cm}
		{\large \today \par}
	\end{center}
\end{titlepage}

\tableofcontents
\newpage

%% All content of this work
\newpage
\section{Aufgabenstellung}

Als Aufgabe des Moduls \textbf{Lerntechniken} erstelle ich eine Visualisierung für ein komplexes Thema aus dem eigenen Studiengang auf elektronischen Medien.
Als Thematik habe ich mich für die Bearbeitung des Themas \textbf{Analysis Tools} mit dem Schwerpunkt \textbf{Big-Oh Notation} aus dem Modul \textbf{\glspl{algorithmus} und \glspl{datenstruktur} 1} entschieden.
\section{Übersicht der Thematik}
Das folgende Mindmap \emph{Abbildung \ref{fig:mindmap-thematik}} soll grob die behandelten und prüfungsrelevanten Themen visualisieren.

%% Beginn des Mindmaps
\begin{figure}[!h]
\centering
	\begin{tikzpicture}
  		\path[mindmap,concept color={rgb:red,1;green,2;blue,5},text=white]
    	node[concept] {Analysis Tools} [clockwise from=-30]
    	child[concept] { 
    		node[concept] {Empirische Analyse}
    	}
    	child[concept, thick] { 
    		node[concept] {Mathematische Funktionen} 
    	}
    	child[concept, thick] { 
    		node[concept] {Big-Oh Notation} 
    	};   
	\end{tikzpicture}
	
\caption{Mindmap über die behandelte Thematik in dieser Arbeit.}
\label{fig:mindmap-thematik}
\end{figure}

Für die Modulprüfung ist es notwendig, die \textbf{O-Abschätzung} durchführen zu können. Als gegeben im Test ist das Zeitverhalten eines Algorithmus anzusehen. In der Klausur vom Frühlingssemester 2015 machte diese Aufgabe \textbf{\nicefrac{1}{9} der Gesamtpunktzahl} aus und ist somit eines der Hauptthemen.
\newpage
\section{Einführung}

\subsection*{Geschichte für die Erläuterung des Begriffs ''Analyse Tool''}
Vor vielen Jahren erhielt Archimedes einen kniffligen Auftrag. König Hieron II. von Syrakus bat ihn herauszufinden, ob seine Krone tatsächlich aus purem Gold bestand.

\par \medskip

Der König hatte einem Goldschmied einen grossen Goldklumpen gegeben, aus dem dieser eine Krone fertigen sollte. Als das Schmuckstück fertig war, versicherte der Schmied dem König, dass die Krone aus reinem Gold war. Doch der Herrscher hegte einen Verdacht: Hatte der Schmied heimlich Silber beigemischt, um einen Teil des wertvolleren Goldes für sich zu behalten?

\par \medskip

Die zündende Idee kam Archimedes der Legende nach, als er in eine randvolle Badewanne stieg und das Wasser überschwappte. Er begriff, dass sein Körper eine ganz bestimmte Menge Wasser verdrängt hatte. Dieser Geistesblitz überwältigte Archimedes angelblich so sehr, dass er vor Freude aus der Wanne sprang und laut ''Heureka, heureka'' geschrieen haben soll. Das ist griechisch und bedeutet ''Ich hab's gefunden!''
\par \medskip
\textbf{Was Archimedes entdeckt hat war ein Analyse Tool, welches in Kombination mit einer simplen Skala, entscheiden konnte, ob die Krone des Königs aus purem Gold ist.}

\subsection*{Bezug zu unserem Thema ''Analyse Tools''}
Bei der Entwicklung von Programmen legen wir einen grossen Wert darauf ''guten'' Code zu schreiben. Wobei wir unter Code \glspl{algorithmus} und \glspl{datenstruktur} verstehen. Aber um solche \glspl{algorithmus} und \glspl{datenstruktur} als ''gut'' bezeichnen zu können, benötigen wir prä­zis Analyse-Werkzeuge. 

\par \medskip

Unser Ziel ist es die Laufzeit eines \gls{algorithmus} und die Operationen auf der \gls{datenstruktur} inklusive der Speicherverwendung chrakteriisieren zu können. Die Laufzeit ist hierbei die zentrale Masseinheit für einen ''guten'' \gls{algorithmus}. Diese Laufzeit ist generell von der zu verarbeitenden Datenmenge abhängig. Jedoch, aber auch von weiteren Faktoren, wie CPU-Belastung, Hardware, Betriebssystem, Programmiersprache, etc \dots Wir konzentrieren uns im Weiteren ausschliesslich auf die Laufzeit des \gls{algorithmus} in Abhängigkeit zur Grösse der Eingabe (Input). Diese Gegebenheit erlaubt es uns, eine handvoll mathematischer Funktionen für die Analyse des \gls{algorithmus} zu verwenden.
$$Funktion(Input) = Laufzeit\ abhängig\ von\ der\ Inputgrösse$$

\section{Empirische Analyse}

Eine Möglichkeit die Effizienz eines \gls{algorithmus} zu ermitteln, ist es diesen \gls{algorithmus} zu implementieren und experimentell durch das Ausführen zu messen. Beispielsweise mit den folgenden Code-Fragmenten \ref{lst:code-fragment-1} und \ref{lst:code-fragment-2}.

\par \medskip

\renewcommand{\lstlistingname}{Code Fragment}% Listing -> Code Fragment

\begin{lstlisting}[captionpos=b,caption={Typische Lösung für die Zeitmessung eines Algorithmus in Java},label={lst:code-fragment-1}]
long startTime = System.currentTimeMillis();
// zu testender Algorithmus ...
long elapsedTime = System.currentTimeMillis() - startTime;   
\end{lstlisting}

\par \medskip

\begin{lstlisting}[captionpos=b,caption={Zeitmessung eines Algorithmus in Java mit nanoTime()},label={lst:code-fragment-2}]
long startTime = System.nanoTime();
// zu testender Algorithmus ...
long estimatedTime = System.nanoTime() - startTime;
\end{lstlisting}

\subsection*{Probleme bzw. Grenzen der empirischen Analyse}
Das Resultat der Messung wird von Computer zu Computer und schlimmer von Ausführung zu Ausführung sich unterscheiden. Daher ist das Vergleichen von zwei \glspl{algorithmus} äusserst schwierig. Insbesondere wird der vollständig Implementierte \gls{algorithmus} vorausgesetzt und das ausprobieren von unendlich vielen Inputgrössen benötigt unendlich viel Zeit, was wir nicht aufbringen können.
\input{contents/05_mathematische-funktionen}
\input{contents/06_big-oh-notation}

%Print the glossary on a new page
\newpage
\printglossaries

\end{document}