%%%%%%%%%%%%%%%%%%%%%%%%%%%%%%%%%%%%%%%%%%%%%%%%%%%%%%%%%%%%%%%%%%%%
%% Summary of Analysis Tools in Data Structues & Algorithms in Java.
%% =================================================================
%% filename: 	analysis-tools.tex
%% created:		20.04.2016
%% -----------------------------------------------------------------
%% (C) 2016 by 	Nikola Stanković
%%				University of Applied Sciences Rapperswil
%%				Switzerland
%%%%%%%%%%%%%%%%%%%%%%%%%%%%%%%%%%%%%%%%%%%%%%%%%%%%%%%%%%%%%%%%%%%%

\documentclass[a4paper,12pt]{scrartcl}

%% Load packages
%% ========
\usepackage[utf8]{inputenc}	% this is needed for umlauts
\usepackage[ngerman]{babel} % this is needed for umlauts
\usepackage[T1]{fontenc}    % this is needed for correct output of umlauts in pdf
\usepackage{lmodern}
\usepackage[]{setspace}
\usepackage{amsmath}
\usepackage{amssymb}
\usepackage{geometry}
\usepackage{graphicx}
\usepackage{hyperref}		% this is needed for the hyper references
\usepackage{nicefrac}		% this is needed for the skewed fractions

\onehalfspacing

%% Glossary
\usepackage[xindy]{glossaries} 
\newglossaryentry{algorithmus}
{
	name={Algorithmus},
	description={ist eine Schritt-für-Schritt Anleitung für die Ausführung einer Aufgabe, bzw. Lösung einer Problemstellung, in einer begrenzten Dauer},
	plural={Algorithmen}	
}

\newglossaryentry{datenstruktur}
{
	name={Datenstruktur},
	description={ist eine systematische Art und Weise, um Daten zu organisieren und darauf zuzugreifen},
	plural={Datenstrukturen}	
}
\makeglossaries

%% For drawing the mindmap
\usepackage{tikz}
\usetikzlibrary{mindmap,trees}

\geometry{papersize={210mm,297mm},total={160mm,240mm},top=31mm,bindingoffset=15mm}

%% Own variables
%% ========
\newcommand{\documenttitel}{Analysis Tools}
\newcommand{\documentsubtitel}{Algorithmen und Datenstrukturen 1}
\newcommand{\documentauthors}{Nikola Stanković}

%% PDF meta informations
%% ========
\pdfinfo{
   /Author (\documentauthors)
   /Title  (\documenttitel)
   /CreationDate (D:20040502195600)
   /Subject (Algorithmen und Datenstrukturen 1, Analysis Tools)
   /Keywords (AD1;Analysis-Tools;HSR)
}

%% For header and footer
\usepackage{scrpage2}\pagestyle{scrheadings} 
\ihead{\documenttitel}
\ohead{Inhaltsverzeichnis}

\begin{document}

\author{
  Stanković, Nikola
}


\begin{titlepage}
	\centering
	\includegraphics[width=0.5\textwidth]{images/HSR_Logo_CMYK.jpg}\par\vspace{1cm}
	\vspace{1cm}
	{\huge \bfseries \line(1,0){250} \\ \documenttitel \\ \line(1,0){250} \par}
	\vspace{0.5cm}	
	{\documentsubtitel \par}
	\vspace{1cm}
	{\Large \bfseries \documentauthors \par}

	\vfill

	% Bottom of the title page
	{\large Hochschule für Technik Rapperswil \\ \today\par}
\end{titlepage}

\tableofcontents
\newpage

%% All content of this work
\input{01_einfuehrung}
\section{Übersicht der Thematik}
Das folgende Mindmap \emph{Abbildung \ref{fig:mindmap-thematik}} soll grob die behandelten und prüfungsrelevanten Themen visualisieren.

%% Beginn des Mindmaps
\begin{figure}[!h]
\centering
	\begin{tikzpicture}
  		\path[mindmap,concept color={rgb:red,1;green,2;blue,5},text=white]
    	node[concept] {Analysis Tools} [clockwise from=-30]
    	child[concept] { 
    		node[concept] {Empirische Analyse}
    	}
    	child[concept, thick] { 
    		node[concept] {Mathematische Funktionen} 
    	}
    	child[concept, thick] { 
    		node[concept] {Big-Oh Notation} 
    	};   
	\end{tikzpicture}
	
\caption{Mindmap über die behandelte Thematik in dieser Arbeit.}
\label{fig:mindmap-thematik}
\end{figure}

Für die Modulprüfung ist es notwendig, die \textbf{O-Abschätzung} durchführen zu können. Als gegeben im Test ist das Zeitverhalten eines Algorithmus anzusehen. In der Klausur vom Frühlingssemester 2015 machte diese Aufgabe \textbf{\nicefrac{1}{9} der Gesamtpunktzahl} aus und ist somit eines der Hauptthemen.
\input{03_empirische-analyse}
\newpage
\section{Übliche Mathematische Funktionen}
\newpage
\section{Big-Oh Notation}

%Print the glossary on a new page
\newpage
\printglossaries

\end{document}